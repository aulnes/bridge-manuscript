\documentclass{article}
\usepackage{graphicx} % Required for inserting images
\usepackage{hyperref}
\title{Questions}

\author{Xun Zhang \quad \quad Bingsheng Zhang \\ 
Zhejiang University, CHN \\
22221024@zju.edu.cn \quad bingsheng@zju.edu.cn}
\date{March 19, 2024}

\begin{document}

\maketitle

\section{Questions about zkBridge}

\begin{itemize}
    \item Q: Does blkH\_r-1 contain a Merkle tree of the public keys used to verify the signatures in the proof for blkH\_r? This would allow us to keep everything succinct.
    
    A: \textbf{YES}(according to Cosmos code). 
    
    \textit{"In Cosmos, each block header contains about 128 EdDSA signatures (on Curve25519), Merkle roots for transactions and states, along with other metadata, where 32 top signatures are required to achieve super-majority stakes."}(sec. 6.1)

    See it in tendermint pkg: \href{https://pkg.go.dev/github.com/tendermint/tendermint/types#Header.NextValidatorsHash}{code}


    $\mathsf{ValidatorsHash \quad \quad tmbytes.HexBytes \quad  `json:"validators\_hash"`   \quad validators for the current block}
    \\
	\mathsf{NextValidatorsHash \quad \quad tmbytes.HexBytes \quad `json:"next\_validators_hash"`\quad validators for the next block}$

    \item Q: Should we reuse existing building blocks or do we want to design our own scheme?

    A: I think it depends on the blockchains we want to bridge. Since there are many blockchains use diverse signature schemes, the performance requirements for the ZK-SNARK are also different. Even in some scenarios, such as bridging from Ethereum to other EVM-compatible blockchains, there is no need to use zksnark(see sec. 6.4).
\end{itemize}
        

\end{document}
