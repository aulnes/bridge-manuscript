\documentclass{article}
\usepackage{graphicx} % Required for inserting images
\usepackage{float}
\title{bridge survey and benchmark}

\author{Xun Zhang \quad \quad Bingsheng Zhang \\ 
Zhejiang University, CHN \\
22221024@zju.edu.cn \quad bingsheng@zju.edu.cn}
\date{March 26, 2024}

\begin{document}

\maketitle

\section{Bridge uses zk-proofs}

There are severe cross-chain bridge based on zk-proof techniques:


\textbf{1. Succinct by Succinct Labs} 
Succinct Labs has developed a system that allows for a trust-minimised connection between Gnosis and Ethereum 2.0, a proof-of-stake consensus blockchain. The system uses SNARKS to efficiently verify the validity of consensus proofs on the Gnosis chain.

The Ethereum 2.0 network has a committee of 512 validators randomly chosen every 27 hours and is responsible for signing every block header during that period. If at least 2/3 of the validators sign a given block header, the state of the Ethereum network is considered valid. 

The verification process of Ethereum mainly includes the verification of the following contents: the Merkle proof of the block header; the Merkle proof of the validators in the synchronization committee; the BLS signature of the correct rotation of the synchronization committee, etc. The core idea here is to use zk-SNARKs (Groth16) to generate constant-size validity proofs that can be efficiently verified on-chain on Gnosis.


\textbf{2. zkIBC by Electron Labs}
Specifically, zkIBC hopes to simulate the Inter Blockchain Communication Protocol (IBC), the trustless communication protocol used by the Cosmos sovereign chain, and expand its use to Ethereum. zkIBC uses ZK-SNARKs for light client status verification, quickly proves transactions on Ethereum, and keeps up with the block time of the Tendermint consensus chain.

However, using a light client from the Cosmos SDK on Ethereum presents some challenges. The Tendermint light client used in the Cosmos SDK operates on the Ed25519 curve, which is not supported natively on the Ethereum blockchain. This makes it expensive and inefficient to verify Ed25519 signatures on Ethereum’s BN254 curve. Electron Labs plans to solve this problem by creating a system based on a zkSNARK, which can generate a proof of signature validity off-chain and only verify the proof on the Ethereum chain.


The current testnet requires waiting approximately 20–30 minutes for finality, which includes Goerli network finality (15–20 minutes), ZK-Proof generation (5–8 minutes), Near chain minting (10–20 seconds) .


\textbf{3. zkBridge by BerkleyRDI}
The construction we want to follow. The main innovations of zkBridge are:

· deVirgo: Uses a distributed approach to generate ZK-SNARK proofs without trust assumptions. The deVirgo method greatly improves the time to generate ZK-SNARK proofs off-chain by splitting the calculation work and allocating it to more devices.

· Recursive proof: In order to reduce on-chain costs, zkBridge uses recursive proof. Through two recursions, the size of the ZK-SNARK proof is compressed to about 131 bytes. The first step is to generate deVirgo proofs, and the second step is to use the Groth16 proof generator for compression. The Groth16 verifier generates integrity proofs of executing deVirgo circuits.

· Batch processing: zkBridge implements a block header update contract, which takes the block height as input and returns the corresponding block header. However, zkBridge does not call the update contract when each new block is generated. The prover can first collect N block headers to generate a single proof. The N value can be set. The larger N, the longer the user waiting time but the lower the system operating cost.

Currently, zkBridge has implemented an instance of Cosmos Client on Ethereum using Solidity. According to tests, it can generate a ZK-SNARK proof of the Cosmos Zone block header in \textbf{2} minutes.



\section{Comparison of bridge}

We compare the existing zero-knowledge bridges from diverse dimensions, including architecture, proof systems and their communication complexity. See it in Table. 1.

\begin{table}[H]
    \centering
    \begin{tabular}{p{3cm}p{3cm}p{3cm}p{3cm}} \hline 
                        & Succinct labs & zkIBC &  zkBridge\\ \hline
         Architecture   & Application specific & Application specific &Application agnostic \\ \hline
         Zk-proofs & Groth16 &Groth16 & deVirgo, Groth16 \\ \hline

         Communication Complexity & O(1) & O(1) per batch of signatures &
         $O(N\log_2R)$ 
         
         N: number of machines 
         
         R: gates per layer\\ \hline 
    \end{tabular}
    \caption{Comparison of bridges}
    \label{tab:my_label}
\end{table}

And we also do a survey about the proof generation cost in these bridge schemes. In order to clarify this issue, we conducted investigations from verification task, constraint size, proof generation time and other aspects. And the cost of verifying the proof is also presented.
See it in Table. 2.


\begin{table}[H]
    \centering
    \begin{tabular}{p{2cm}p{4cm}p{4cm}p{4cm}} \hline 
                        & Succinct labs & zkIBC &  zkBridge\\ \hline
         Verification Task   
         & Sync Committee: 512 high stake signatures

         Verification of Ethereum BLS signatures on Gnosis
         &  Verification of EdDSA Curve25519 signatures on Ethereum
         
         Validators: 32 high stake signatures
         & Verification of EdDSA Curve25519 signatures on Ethereum
         
         Validators: 32 high stake signatures
         
         Computation distributed across relay networks
         \\ \hline
         Constraint Size & 
         Sync Committee rotation: 68M constraints

         Verify signed header: 21M constraints
         &
         2.5M constraints for every Ed25519 signature
         
         & 2.5M constraints for every Ed25519 signature\\ \hline

         Proof Generation Time & 
         \~180 secs
         & 
         \~300 secs per batch proof of 32 signatures.

         With parallelism: $\textless$ 7 secs
         &
         \~18.21 secs per proof of 32 signatures.
         \\ \hline 
         Gas Cost &
         226k gas per signature
         &
         300k gas per batch
         &
         227k gas(constant because of Groth16) \\  \hline
    \end{tabular}
    \caption{Comparison of cost}
    \label{tab:my_label}
\end{table}




\section{Comparison of Groth16 and ATMS}

We first compare the on-chain cost of Groth16 verifying with Ad-hoc Threshold Multi-Signatures(atms) over Cardano.

First, there are some bls12-381 curve operation costs, which has been estimated Kenneth that show how expensive each function is. Follwing is the table of cost, where $x$ is the size of the input in bits divided by 64. See it in Table. 3.


\begin{table}[H]
    \centering
    \begin{tabular}{p{5cm}|p{7cm}} \hline
         Function name & cost(cpu units)  \\ \hline
         bls12\_381\_G1\_compress&   3341914 \\ \hline
         bls12\_381\_G1\_uncompress&   16511372 \\ \hline
         bls12\_381\_G1\_add&   1046420 \\ \hline
         bls12\_381\_G1\_equal&   545063 \\ \hline
         bls12\_381\_G1\_hashToCurve&   66311195 + 23097*x  \\ \hline
         bls12\_381\_G1\_mul         &94607019 + 87060*x \\ \hline
         bls12\_381\_G1\_neg         &292890 \\ \hline
         bls12\_381\_G2\_compress&   3948421 \\ \hline
         bls12\_381\_G2\_uncompress&   33114723 \\ \hline
         bls12\_381\_G2\_add&   2359410 \\ \hline
         bls12\_381\_G2\_equal&   1102635 \\ \hline
         bls12\_381\_G2\_hashToCurve&   204557793 + 23271*x  \\ \hline
         bls12\_381\_G2\_mul         &190191402 + 85902*x \\ \hline
         bls12\_381\_G1\_neg         &307813 \\ \hline
         bls12\_381\_GT\_finalVerify & 388656972 \\ \hline
         bls12\_381\_GT\_millerLoop  & 402099373 \\ \hline
         bls12\_381\_GT\_mul         & 2533975   \\ \hline
         blake2b\_256                & 358499 + 10186*x (521475, with x = 16) \\ \hline
         addInteger                  &85664 + 712*max(x,y) (88512, with x = y = 4) \\ \hline
         multiplyInteger             &1000 + 55553*(x+y) (641924, with x = y = 4, and we include the price of modular reduction, as we need one per mult)  \\ \hline
         divideInteger            &if x$>$y then  809015 + 577*x*y else  196500 \\ \hline
         modInteger                      & 196500 \\ \hline
         expInteger                      & We estimate 32 mults and adds (23373952) \\ \hline
    \end{tabular}
    \caption{Function Cost}
    \label{tab:my_label}
\end{table}


Then we can make a back-of-the-envelope computations to see how feasible it is to verify SNARKsn or a atms on main-net.

Here is the comparison table of the cost, where $M$ is the total number of signers(or the committee size), and $N$ is the number of non-signers. See it in Table. 4.

\begin{table}[H]
    \centering
    \begin{tabular}{p{5cm}|p{4cm}|p{4cm}} \hline
         Funtion & Groth16 Verification & ATMS Verification  \\ \hline
         bls12\_381\_G1\_uncompress&  4 & $N$ \\ \hline
         bls12\_381\_G2\_uncompress&  4 & 0 \\ \hline
         bls12\_381\_G1\_mul&  1 & 0 \\ \hline
         bls12\_381\_G1\_add&  1 & $N+1$ \\ \hline
         bls12\_381\_GT\_millerLoop  & 4 & 2 \\ \hline
         bls12\_381\_GT\_mul         & 2 & 0 \\ \hline
         bls12\_381\_GT\_finalVerify & 1 & 1 \\ \hline
         blake2b\_256 & 0 &  $\log_2M *N$  \\ \hline
         Total & 2,299,066,153 & 1,914,978,116
         
         ($M=100, N=34$) \\ \hline
    \end{tabular}
    \caption{Verification Cost}
    \label{tab:my_label}
\end{table}

When the number of signers incrase, the cost of ATMS behave like Table. 5.

\begin{table}[H]
    \centering
    \begin{tabular}{p{3cm}|p{3cm}|p{3cm}|p{3cm}} \hline
         signers number& non-signers number& cert sizr &total cost  \\ \hline
         100&  34& 3284 &1,914,978,116 \\ \hline
         200&  67& 6280 &2,649,784,802 \\ \hline
         300&  100& 8988&3,419,008,838 \\ \hline
         400&  134& 12132&4,175,545,116 \\ \hline
         500&  167& 14712&4,909,830,327 \\ \hline
         600&  200& 18796&5,748,410,538 \\ \hline
         700&  234& 21460&6,522,676,966 \\ \hline
         800&  267& 24104&7,274,170,852 \\ \hline
    \end{tabular}
    \caption{ATMS Verification Cost}
    \label{tab:my_label}
\end{table}



\section{Proving Time}

we also benchmark the proving time of EdDSA signature over curve 25519, using Halo2 as proof system.

We use the code from Ayush Shukla, and the result are follwing:

\begin{table}[H]
    \centering
    \begin{tabular}{c|c|c|c|c|c|c|c} \hline
      degree&num\_advice&	num\_lookup&	num\_fixed&	limb\_bits&		proof\_time&	proof\_size&	verify\_time
    \\ \hline
     19&	1	&1	&1&	88&		11.9s&	1920&	67.20ms \\ \hline
     18&	2&	1	&1&	88&	8.7s&	3200	&63.43ms \\ \hline
     17&	4&	1&	1&	88&	5.7s&	5632&	44.95ms \\ \hline
     16&	8&	2&	1&	90&		5.5s&	10976&	47.02ms \\ \hline
     15	&17	&3	&1		&90&	6.0s&	22688&	40.15ms \\ \hline
     14	&34	&6	&1	&	91	&7.3s&	46240&	57.24ms \\ \hline
     13	&68	&	12&	1&	88&		9.6s&	94048&	76.64ms \\ \hline
     12	&139&	24	&2	&	88&	14.4s&	192736&	126.37ms \\ \hline
     11	&291&	53	&4	&	88&		35.8s&	416800	&189.40ms \\ \hline


    \end{tabular}
    \caption{EdDSA signature benchmarks}
    \label{tab:my_label}
\end{table}


And compare with the zero-knowledge bridges we mentioned before, it will be competitive just using halo2 trivially. The proof generation phase may cost about 200 secs, which is 10x slower than zkBridge.

\end{document}
