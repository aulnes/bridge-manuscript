\documentclass{article}
\usepackage{graphicx} % Required for inserting images
\usepackage{float}
\usepackage{amsmath}
\usepackage{amsfonts}


\title{24.05.21 SNARK-based ATMS}
\author{Xun Zhang \quad \quad Bingsheng Zhang \\ 
Zhejiang University, CHN \\
22221024@zju.edu.cn \quad bingsheng@zju.edu.cn}

\date{May 21 2024}
\begin{document}

\maketitle

\section{SNARK-based ATMS}


\subsection{Comparison between Mithril and ATMS}

Use Code from Inigo.


We provide a quick comparison to see what are the differences and Similarities between Mithril
and implemented ATMS(SNARK version).


\begin{table}[H]
    \centering
    \begin{tabular}{p{4cm}|p{3cm}|p{4cm}}    \hline
         &Mithril & ATMS(SNARK version) \\ \hline
         core signature scheme& BLS &Schnorr \\ \hline
         threshold& stake sum & valid signature num \\ \hline
         VK commitment& Merkle tree& Hash \\ \hline
         hash function& blake2b & Rescue \\ \hline
         eligibility check & Yes &No \\ \hline
         curve& bls12-381 &bls12-381 \\ \hline
    \end{tabular}
    \caption{Mithril and ATMS Comparison}
    \label{tab:my_label}
\end{table}


\subsection{Workflow of SNARK-based ATMS}
Assume that there exists $n$ committee members, and the required threshold is $t$.

\textbf{Step 1.} Each individual signer proceeds the [$keygen$]
The function generates $(sk_i, pk_i)$ as the keypair for the $signer_i$.

\textbf{Step 2.} Signers share their public keys with the registration authority:
\begin{itemize}
    \item The role of the registration authority is simply to commit to all public keys of the committee in a Merkle Tree (MT).
    \item The Registration Authority can be a Plutus script, a trusted party, or be distributed amongst the committee members.
    \item The reason why it needs to be 'trusted' is because it can exclude certain participants, or include several keys it owns.
\end{itemize}

  
\textbf{Step 3.} Once all registration requests have been submitted with their corresponding public keys, $pks = [pk_1, ..., pk_n]$, the aggregated public key is created $avk = H(pk_1, \ldots, pk_n)$.

\textbf{Step 4.} Individual parties generate their single signature with [$sign$] and send the signature to aggregator (does not need to be trusted).
Individual signatures should be verifiable with [$verify$]

\textbf{Step 5.} Aggregator receives the single signatures. It collects at least threshold-many valid signatures as the aggregate signature.

\textbf{Step 6.} Once the aggregator receives at least $t$ valid signatures $sig_1, ..., sig_t$ it proceeds to generate the SNARK. In particular, it proves that:

\begin{itemize}
    \item There exists $t'$ valid and distinct signatures, $sig_1, ..., sig_t$ for public keys $pk_1, ..., pk_t$ and message $msg$.
    \item The hash of all $t$ public keys, together with some other set of keys, results in the corresponding $avk$.
\end{itemize}












\subsection{Benchmark of SNARK-based ATMS}



We bench the SNARK-based ATMS by Criterion crate.

k represents the degree of polynomial(which means the number of rows is about $2^k$). n represents the number of parties. And th represents the threshold of signatures.


\begin{table}[H]
    \centering
    \begin{tabular}{c|c} \hline
        Setting & Proving Time \\ \hline
         k = 14, n = 6, th = 3& 1.2579s  \\ \hline
         k = 15, n = 9, th = 6& 2.3281s \\ \hline
         k = 15, n = 9, th = 8& 2.3310s \\ \hline
         k = 16, n = 15, th = 15& 4.3034s \\ \hline
         k = 16, n = 21, th = 14& 4.3301s \\ \hline
         k = 17, n = 21, th = 17& 8.0298s \\ \hline
         k = 17, n = 42, th = 28& 8.1542s \\ \hline
    \end{tabular}
    \caption{Proving time of ATMS}
    \label{tab:my_label}
\end{table}

\begin{table}[H]
    \centering
    \begin{tabular}{c|c} \hline
        Setting & Proving Time \\ \hline
         k = 19, n = 102, th = 72& 28.761s  \\ \hline
         k = 22, n = 2001, th = 1602 & $\approx$ 200s \\ \hline
    \end{tabular}
    \caption{Proving Time of Real Situations}
    \label{tab:my_label}
\end{table}


Note that the benchmark of last setting(k = 22, n = 2001, th = 1602) is computed by hand, since my computer does not support the long running time of function.




\section{Rescue and Poseidon}

Since Mithril is implemented with Poseidon and SNARK-based ATMS is on Rescue, it is necessary to compare Poseidon and Rescue.

The Poseidon functions implemented in halo2 and halo2-lib are both \textbf{Poseidon128}, which means it has a \textbf{128-bit} security level.

We provide a quick comparison between Poseidon128 and Rescue in implementation:

\begin{table}[H]
    \centering
    \begin{tabular}{c|c|c} \hline
         &Poseidon128& Rescue  \\ \hline
         security level& 128bit&128bit \\ \hline
         S-box & $x^5$& $x^5$\&$x^{1/5}$ \\ \hline
         Support curve& BN/BLS/Ed& BN/BLS/Ed \\ \hline
         Width&  3,9,12 &4 \\ \hline
         Capacity& 1 & 1 \\ \hline
         $R_f$ &8 & 12 \\ \hline
         $R_p$&57& - \\ \hline
    \end{tabular}
    \caption{Comparison of Poseidon128 and Rescue}
    \label{tab:my_label}
\end{table}



\subsection{Performance}

The performance we benched are in quite different settings, is for reference only. 


\subsection{Poseidon Benchmark}

The field of halo2 official Poseidon is \textbf{pallas/vesta}.

And the \textbf{rate = width - 1}. All the results only do \textbf{one-time} permutation, because the input length is strictly equals to the rate.

It should be noted that the halo2 official implementation used a Blake2b transcript and run a real workflow(including generating PK/VK, proving and verification). So the time it cost may be longer than Rescue.

\begin{table}[H]
    \centering
    \begin{tabular}{c|c|c} \hline
         &Proving & Verification  \\ \hline
         width = 3& 58.371 ms & 3.3824 ms \\ \hline
         width = 9& 106.29 ms & 3.7644 ms \\ \hline
         width = 12& 139.08 ms& 3.9400 ms \\ \hline
    \end{tabular}
    \caption{Halo2 Official Poseidon Benchmark}
    \label{tab:my_label}
\end{table}


The halo2-lib crate uses a optimized Poseidon implementation described in Supplementary Material Section B of https://eprint.iacr.org/2019/458.pdf, aka Poseidon paper(full version). This involves some further computation of optimized constants and sparse MDS matrices beyond what the Scroll PoseidonSpec generates.

The rate is fixed to \textbf{2}, and the capacity is fixed to \textbf{1}.
Which means that the arity(in the context of tree hashing) of this function is \textbf{2}.

And the field of halo2-lib Poseidon is \textbf{BN254}.

\begin{table}[H]
    \centering
    \begin{tabular}{c|c|c|c|c} \hline
         &Generate VK& Generate PK & Proving &Verification \\ \hline
    maxlen = 0  & 59.107ms& 12.434ms & 162.234ms& 2.584ms \\ \hline
    maxlen = 2*2 &81.841ms& 21.711ms& 293.331ms& 3.925ms \\ \hline
    maxlen = 2*5 &127.393ms& 25.423ms& 370.451ms& 4.608ms \\ \hline
    maxlen = 2*2+1&94.544ms& 24.769ms& 312.785ms& 4.069ms \\ \hline
    maxlen = 2*5+1&125.951ms& 41.236ms& 349.318ms& 4.319ms \\ \hline
    \end{tabular}
    \caption{Halo2-lib Poseidon Benchmark}
    \label{tab:my_label}
\end{table}

The halo2-lib crate uses a $\mathbf{base\_test().k().bench\_builder()}$ method to bench the function. It runs keygen, real prover, and verifier by providing a closure that uses a 'builder' and 'RangeChip'.



We can do very rough calculation to see the cost of one permutation under the setting \textbf{rate = 2}.

The proving time of a permutation is about:
\begin{itemize}
    \item $(370.451ms - 293.331ms) / (5 - 2) = 25.7066ms$ or
    \item $(312.785ms - 293.331ms) / 1 = 19.454ms$
\end{itemize}


\subsection{Rescue Benchmark}

The benchmark of Rescue is running under follwing settings:

\begin{itemize}
    \item \textbf{Curve}: BLS12-381
    \item \textbf{RATE}: 3
\end{itemize}


\textbf{Note}: this benchmark is using $\mathbf{MockProver::run}$ method, with parameter \textbf{k = 10}. Thus the real proving time of Rescue hash function will be \textbf{much lower}.

\begin{table}[H]
    \centering
    \begin{tabular}{c|c|c|c|c} \hline
         iteration& width & operation & total time & time per permutation \\ \hline
         4& 4& 12-to-3 &36.269ms & 2.2668ms \\ \hline
    \end{tabular}
    \caption{Rescue Hash Benchmark}
    \label{tab:my_label}
\end{table}


This benchmark uses \textbf{12} BLS12-381 scalar filed element as in put , and get a output element sequence of length \textbf{3}. Of which the first element is the hash result.

A more detailed description is, the benched function "absorbs" the input elements \textbf{12} times, and "squeezes" an output of length \textbf{3}.

Since the rate of Rescue is 3, take \textbf{3} field element as a input group, and straightly do a addition operation on \textbf{state}. That means the benched function do \textbf{4} permutations in total.

\section{Discussion}

\begin{enumerate}
    \item The SNARK-based ATMS uses Jubjub for in-circuit elliptic curve operations since it provides efficient EC operations within the proof. Jubjub is s an elliptic curve of the twisted Edward's form:
    $$E_{d}: -u^2 + v^2 = 1 + du^2v^2.$$
    Define the Jubjub curve over the field $\mathbb{F}_q$ where $q$ is represented in hexadecimal as follows:
    $$q = 0x73eda753299d7d483339d80809a1d80553bda402fffe5bfeffffffff00000001$$seSet the Jubjub curve as the embedded curve of BLS12-381.
    Meaning that, Jubjub curve is defined over a prime which is also the prime that defines the scalar field of BLS12-381.


    \item Since the we only want to get a commitment of public keys, there is "no need to of a merkle tree inside a SNARK, with a hash it is sufficient". So basically, $avk = H(pk_1, pk_2, ..., pk_n)$.

    \item A little problem: The proof must include only threshold-many valid signatures even if the prover has more valid signatures.

\end{enumerate}



\end{document}
