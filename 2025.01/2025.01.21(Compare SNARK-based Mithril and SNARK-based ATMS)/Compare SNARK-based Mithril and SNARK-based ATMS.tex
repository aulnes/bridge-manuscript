\documentclass{article}
\usepackage{graphicx} % Required for inserting images

\title{25.01.21 Compare SNARK-based Mithril and SNARK-based ATMS}
\author{Xun Zhang \quad \quad Wuyun Siqin \quad \quad Bingsheng Zhang \\ 
Zhejiang University, CHN \\
22221024@zju.edu.cn \quad 3210101763@zju.edu.cn \quad bingsheng@zju.edu.cn}
\date{January 21 2025}
\begin{document}

\maketitle

\section{Mithril and ATMS}

\textbf{ATMS},
from the \textit{Proof-of-Stake Sidechains} paper, is a signature scheme designed for sidechains' certificate:

\vspace{0.3cm}

\textit{An important feature of our construction is merged-staking
that prevents "goldfinger" attacks against a sidechain that is only carrying a small amount of stake. An important technique for pegging chains that we use in our construction is cross-chain certification which is facilitated by a novel cryptographic primitive we introduce called ad-hoc threshold multisignatures (ATMS) which may be of independent interest. We show how ATMS can be securely instantiated by regular and aggregate digital signatures as well as succinct arguments of knowledge such as STARKs and bulletproofs with varying degrees of storage efficiency}

\vspace{0.3cm}

The nodes on the mainchain will get information from sidechains, and the information allow them to authenticate a small number of stakeholders on the sidechains, whom are trusted to present a majority of all stakeholders. The information changes periodly, since the stake distribution is a dynamic value.


All the transactions and authentication information will be packed per "epoch", the paper call it a \textit{cross-chain certification}.


\vspace{0.3cm}


\textbf{Mithril}, from the paper \textit{Mithril: Stake-based Threshold Multisignatures}, is a signature scheme designed for mainchain's certificate:

\vspace{0.3cm}

\textit{We also examine the problem of bootstrapping light clients in Proof of Stake
(PoS) blockchains. The general challenge in this setting is that the client needs
to verify the ledger upon joining the network and that block verification fundamentally depends on stake (unlike an SPV client in the bitcoin setting, that can
simply count the blocks’ aggregate difficulty). As a result, a client bootstrapping
in the PoS setting needs to follow the stake as it moves between accounts to be
in sync over time with the stakeholder distribution and validate all the blocks.
The amount of work to be performed scales linearly with the number of transactions in the ledger which can be extremely large. Using mithril, a different
approach can be followed: instead of verifying transactions, the stakeholders can
issue checkpoints at regular intervals using an STM signature. The client needs
only to verify all checkpoints till the most recent one after which individual
blocks and transactions can be verified sequentially. In this way the operation
becomes linear in the number of checkpoints instead of linear in the number
of transactions. The frequency of the checkpoints can be set to be at regular
intervals.}

\vspace{0.3cm}

In our zero-knowledge bridge context, we refer to the second application in the Mithril paper, that is the quick bootstrapping of Mithril light client(Cardano light client). The Mithril protocol provides Blockchain checkpoints periodly, and the light client can verify the Mithril certificates continuously.

\vspace{0.3cm}

We offer a quick comparison of two schemes:

\begin{table}[htbp]
    \centering
    \renewcommand{\arraystretch}{1.1} % 增加行间距
    \setlength{\tabcolsep}{10pt}      % 增加列间距
    \begin{tabular}{c|c|c} \hline
         & \textbf{Mithril} & \textbf{ATMS}   \\ \hline
        core signature  & BLS  & Schnorr/BLS    \\
        aggregation & complex & simple \\
        proving task& heavy & light \\
        target& N/A &mainchain \\
        participation & low percentage & high percentage \\ 
        use case& bootstrapping & cross-chain \\ \hline
    \end{tabular}
    \label{tab:example_benchmark_selected}
\end{table}

The ATMS is used for blockchain's interoperability, but as described in the paper, ATMS signature can only be verified on the mainchain. So if we want to construct a bridge using ATMS, it can only bridge from a sidechain to Cardano mainchain.

While Mithril is not designed for cross-chain, and the "light client" use case of Mithril allow it to be used for bridge. Thus if we construct a bridge using Mithril, it can bridge from Cardano to "any" other blockchain(whether in the same ecosystem or not).

\newpage

\section{ATMS in Schnorr and BLS}

As we discussed in the former documents, when turns to BLS signature scheme, the original proving method(relations in the paper) will be very costly. So we prove the aggregation key and verify the ATMS signature.

Below is our benchmark result of ATMS with product of all keys:


\begin{table}[htbp]
\centering
\begin{tabular}{c|c|c|c}
\hline
\textbf{num} & \textbf{proof\_time} & \textbf{proof\_size} & \textbf{verify\_time} \\ \hline
3 of 6 & 25.27s & 11872 & 13.42ms \\
6 of 9 & 26.81s & 12224 & 12.45ms \\
8 of 9 & 26.59s & 12224 & 12.07ms \\
14 of 21 & 27.51s & 12576 & 14.54ms \\
17 of 21 & 27.45s & 12576 & 13.15ms \\
28 of 42 & 28.80s & 13376 & 11.48ms \\
72 of 102 & 32.13s & 15104 & 13.80ms \\
1602 of 2001 & 160.75s & 80192 & 39.05ms \\ \hline
\end{tabular}
\caption{Proof Time, Proof Size and Verify Time for ATMS in BLS}
\end{table}


And we made a quick comparison of two ATMS proving method, only compare the proving time:




\begin{table}[htbp]
\centering
\begin{tabular}{c|c|c}
\hline
\textbf{num} & \textbf{BLS ATMS} & \textbf{Schnorr ATMS}  \\ \hline
3 of 6 & 25.27s & 1.26s  \\
6 of 9 & 26.81s & 2.33s  \\
8 of 9 & 26.59s & 2.33s \\
14 of 21 & 27.51s & 4.33s  \\
17 of 21 & 27.45s & 8.03s  \\
28 of 42 & 28.80s &8.15s  \\
72 of 102 & 32.13s & 28.76s  \\
1602 of 2001 & 160.75s &  $\approx$ 240s \\ \hline
\end{tabular}
\caption{ATMS Proving Time Comparison}
\end{table}


The reason why BLS ATMS's proving time do not show significant changes in aggregation number is that we did not customize circuit's degree, but it is actually a $O(n)$ algorithm. 

Note that these two versions of ATMS benchmarks are running over very different settings, following is the information:

\begin{table}[htbp]
\centering
\begin{tabular}{c|c|c}
\hline
\textbf{} & \textbf{BLS ATMS} & \textbf{Schnorr ATMS}  \\ \hline

machine & Linux Server & PC \\ \hline
CPU& Intel Xeon Silver 4214 (48 cores) 3.2GHz&Intel Core i5-12500kf 3.6GHz  \\
\hline
RAM & 8 GB& 128 GB \\ \hline
library & halo2-lib & sidechains-zk \\ \hline
curve & bn254& bls12-381/jubjub \\ \hline
hash& Poseidon& Rescue \\ \hline


\end{tabular}
\caption{ATMS Benchmark Settings}
\end{table}

So according to the benchmark settings, the actual performance of two ATMS schemes is close.



\end{document}
